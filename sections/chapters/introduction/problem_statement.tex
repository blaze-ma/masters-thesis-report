A common task of historians is to digitize, parse and categorize historical written records.
One such project us The Middle East Heritage Data Integration Endeavour (MEHDIE) ~\cite{MEHDIE}.
The aim of MEHDIE is to aggregate and align multilingual data generated in the medieval Middle East.

One such dataset is derived from Yaqut al-Hamawi's Dictionary of Countries ~\cite{Yaqut}.
This dataset was created by scanning a manuscript with OCR technology, and then using a rule based approach parsed the
available information.
The methodology and the creation of the dataset is described in ~\cite{YaqutRB}

The exact structure of the final dataset is described in section ~\ref{sec:yagut}.

However, this dataset has some shortcomings.
First, it is available in a non-standard format, making it difficult for subsequent researchers to consume the data.
Second, it is strictly based on the data parsed from the original manuscript.
This however hinders MEHDIE's data alignment initiatives.

This thesis addresses the former issue by experimenting with various knowledge graph representations of the original dataset,
and addresses the latter issue, by using these new representations to create link prediction models to expand the available
information.
