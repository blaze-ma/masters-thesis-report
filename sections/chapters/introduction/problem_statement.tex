A common task of historians is to digitize, parse, and categorize historical written records.
One such project is The Middle East Heritage Data Integration Endeavour (MEHDIE) ~\cite{MEHDIE}.
MEHDIE aims to aggregate and align multilingual data generated in the Medieval Middle East.

One such dataset is derived from Yaqut al-Hamawi's Dictionary of Countries ~\cite{Yaqut}.
This dataset was created by scanning a manuscript with OCR technology and then extracting parsed data from
the text entries using a rule-based model~\cite{YaqutRB}

However, this parsed dataset has some shortcomings.
First, it is only available in a non-standard format, making it difficult for subsequent researchers to consume the data.
Second, it is strictly based on the data parsed from the original manuscript.
This limitation, however, hinders MEHDIE's data alignment initiatives.
This thesis addresses the former issue by experimenting with various knowledge graph representations of the original dataset
. It addresses the latter issue by using these new representations to create link prediction models to expand the available
information.