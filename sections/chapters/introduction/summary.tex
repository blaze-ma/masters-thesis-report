This thesis describes various transformation and prediction tasks executed on the data generated by a 2023 paper ~\cite{YaqutRB} that parsed the medieval gazetteer Kitāb Mu'jam al-Buldān written by Yâqût al-Hamawî~\cite{Yaqut}
First, the various technologies and concepts that form the core of this thesis are introduced.
These concepts include but are not limited to: Knowledge Graphs, Graph Neural Networks, Neo4J~\cite{Neo4j}, Wikidata~\cite{Wikidata} and various
evaluation metrics such as Mean Rank and Hits@K.

Then, the next short section provides details on the original gazetteer and its parsed version.

Having established the base knowledge necessary to follow this thesis, the next sections analyse the initial parsed dataset represented as a graph.
The graph, in its purest form includes only \textit{place} type nodes,  \textit{administrative hierarchical} and \textit{distance} edges.
This analysis includes metrics such as network density, average node degree and clustering.

Then, some initial link prediction experiments were performed using various scoring based models, as implemented in Ampligraph~\cite{ampligraph}.
These approaches are namely: TransE~\cite{TransE}, DistMult~\cite{DistMult}, ComplEx~\cite{ComplEx}, HolE~\cite{HolE} and RotatE~\cite{RotatE}.
However, these first attempts were lackluster  both in terms of statistical performance, and the ability to predict new, true positive links.
Within this section, there's also a dedicated part for triplet candidate discovery strategies, as it is a computationally hard problem.
Various techniques, such as hop limiting and cluster triangles are detailed.

After analysing the shortcomings of the initial models, first, the structure of the graph was pruned.
The graph on which the previous models were trained on, included various ancillary edges and nodes representing data scraped from Wikidata.

Moreover, previously, the graph's ontology did not differentiate between various distances, and treated them equally.
In later version, these edges are binned according to predefined categories.
Second, the hierarchical edges were explicitly defined across all levels.
Third, reverse edges were added.

While the initial models detailed in the previous sections performed marginally better with this new graph representation,
clearly, there was a need to experiment with recent state-of-the-art models.
Therefore, the rest of the thesis uses Neural Bellman-Ford Networks~\cite{NBFNet} as its basis (NBFNet).
The model proposed in the NBFNet paper is currently SOTA~\cite{NBFNetSota} for link prediction using the FB15K-237 dataset .

The next section evaluates the results after training the NBFNet model on the gazetteer dataset.
While the results are significantly better, it is hypothesized that it is possible to reach even better results
by pretraining the model on a similar, but orders of magnitude larger, synthetic dataset.
To create such dataset, the thesis relies on WorldKG~\cite{WorldKG}, a geographical knowledge graph constructed based on Open Street
Maps~\cite{OpenStreetMap} data.

The data found in the WorldKG dataset is used to create a synthetic dataset mimicking Yāqūt's Kitāb Mu'jam al-Buldān,
but utilizing nodes from the entire world.
This synthetic dataset also allows the introduction of specific biases that are not commonly found in the original datasource,
but are of interest to the researchers.
For example: explicit distance edge to the largest settlement, within a certain radius, and the closest geographical neighbor.

Then, the performance of a pretrained NBFNet model is compared to previous experiments, showcasing the improvements.

Finally, in-line with recent developments in the field of machine which happened during the writing of this thesis.
The most successful NBFNet version is recreated, but using Kolmogorov-Arnold Networks~\cite{KAN} (KANs) in place of classic multilayer
perceptron.

The final section of the thesis reiterates the results, and reflects on potential improvements, shortcomings, and future
research possibilities.





