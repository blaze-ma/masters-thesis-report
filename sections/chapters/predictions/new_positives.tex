To generate previously unknown positive triplets, one needs to generate a set of candidates.
A generation strategy is necessary as trying all possible $(h,r,t)$ combinations is computationally expensive
with its cost increasing exponentially with the number of nodes and edge types.

In the case of hierarchical edges, first,
the nodes with missing hierarchical information were selected as the head entities.
Second, all place nodes within 2-hops of the head entity were selected as tail entity candidates.

For the distance-type edges, such a hop limitation would not be beneficial as the kilometer value of the edges ranges from
One kilometer to over a thousand kilometers.

Therefore, it was decided to randomly sample nodes with low centrality measurements.
This approach has the benefit of potentially providing the highest value information.

Predicting a new edge for an entity that only has 1--2 connections to the rest of the graph
increases the average node degree at a larger ratio than the prediction of a new edge for Baghdad, which is
already an extremely central node in the graph.

The Ampligraph~\cite{ampligraph} library exposes several strategies for finding such nodes.
These strategies are:
\begin{itemize}
    \item Entity Frequency
    \item \textbf{Entity connectedness:} Graph Degree, Cluster Coefficient
    \item  \textbf{Local Graph Structures:} Cluster Triangles, Cluster Squares
\end{itemize}


The Local Graph Structures Strategies rely on the idea that well-connected structures are less likely to have missing facts.
In this thesis, the Cluster Triangles approach was used to generate triplet candidates.

\subsection{Hierarchical Triplet Candidates}
The general feedback on the predicted hierarchical candidates was that the model is seemingly able to recognize hierarchical
ordering between two places but struggles to correctly identify the exact level of hierarchy.
A false positive example highlighted in the evaluation was the prediction that
\RL{المراكب} (almrākb), a city in Tunisia was in the province of Tunisia.
However, Tunisia is not actually a province; instead, both Tunisia and \RL{المراكب} are in the province of North Africa.
Upon reviewing the graph, it was found that no node has a \textbf{wd\_P131\_PROVINCE} edge with Tunisia as the tail.

When making this prediction, the model's decision could be followed as:
\begin{verbatim}
weight: 3.33629:
<almrākb, wd_P131_DISTRICT, Tunisia>
\end{verbatim}
This is to be interpreted such that the model's main reason for predicting this triplet to be true is
the known \textbf{wd\_P131\_DISTRICT} edge between the two entities.
This is a rather odd finding, as no similar patterns were found in the training dataset, where the district and
the province tail entity of a head node is the same.


\subsection{Distance Triplet Candidates}
The Distance predictions fail in a similar manner, the general feedback was that the model seems to be able to capture
some form of semantic relation between places.
For example, the highest rated distance predictions are consistently for places that are found in the same
administrative area.
However, the types of distance edge chosen are often wildly inaccurate.
For example, a highly rated false positive triplet is between the Spanish city of Córdoba and the Spanish municipality
of Toledo.
The model, with high certainty predicted that these two entities should be within 5 kilometers of each other, when
in reality they are roughly 230 kilometers away from each other.

The main contributor to this prediction is as follows:
\begin{verbatim}
weight: 3.27499
<Córdoba, DISTANCE_200_500, Toledo>
\end{verbatim}

This is once again an unusual result, as the model predicted a different distance edge based on the existence of
another distance edge between the two places.
It also highlights a bug in the candidate selection logic.
