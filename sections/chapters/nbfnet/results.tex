Unfortunately,
the performance of the trained NBFNet model (table~\ref{tab:nbfnet-res})
on the previously unseen test dataset was comparable to
the performance of the initial KGE models.
It performed slightly worse in terms of MRR, but it achieved the highest score on Hits@10 and Hits@5

\begin{table}[!ht]
    \centering
    \begin{tabular}{|l|l|l|l|l|}
        \hline
        & MRR & Hits@10 & Hits@5 & Hits@1 \\ \hline
        TransE & 0.11   & 0.24 & 0.14 & 0.04 \\ \hline
        RotatE & 0.16   & 0.21 & 0.17 & 0.13 \\ \hline
        HolE & 0.18     &  0.22 & 0.19 & 0.16 \\ \hline
        DistMult & 0.18 & 0.21 & 0.18 & 0.16 \\ \hline
        ComplEx & 0.18  & 0.22 & 0.19 & 0.16 \\ \hline
        NBFNet & 0.17 & \textbf{0.27} & \textbf{0.21} & 0.11 \\ \hline
    \end{tabular}
    \caption{Performance of the KGE and NBFNet models}
    \label{tab:nbfnet-res}
\end{table}

A potential approach to improve the model's performance is discussed in the next chapter.