To construct the synthetic dataset, first, beneficial patterns need to be created.
While it is entirely possible to create a fully connected graph, it would not represent the biases found Kitāb well.
Nodes like Alexandria are central because Yâqût found it important to define places in relation to big, central places.
Second, the local structures in Kitāb KG exist because Yâqût also found it important to define places in relation
to their surroundings.
These ideas may be boiled down into the following simple bullet points:
\begin{itemize}
    \item A nodes's local neighborhood should be well-connected.
    \item Each node should be connected to some central entity.
    \item Central entities should be connected to each other.
\end{itemize}

Following these simple rules, one could create a Kitāb KG-like synthetic dataset with higher density.
In theory the entire WorldKG dataset could be used to create such synthetic dataset, for practical purposes,
in reality the data was heavily down-sampled.

\subsection{Selecting Relevant Nodes}
In WorldKG there are over 1000 node types, most of them irrelevant for this thesis' purpose, since, for example,
there were exceedingly few airports in the medieval arab world.
Instead, the categories discussed in section TODO were mapped to their OSM equivalent.

Then, with Baghdad selected as the centre point, all WorldKG place nodes (of relevant type) were selected in a 2000km
radius, using the following SparQL query:

\begin{verbatim}
\end{verbatim}
