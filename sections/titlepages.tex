\pdfbookmark[0]{English title page}{label:titlepage_en}
\aautitlepage{%
  \englishprojectinfo{
      Exploring Methods for Link Prediction on a Historical Geographic Knowledge Graph
  }{%
    Scientific Theme %theme
  }{%
    Spring Semester of 2024 %project period
  }{%
  }{%
    %list of group members
    Balázs Márk Agárdi
  }{%
    %list of supervisors
    Tomer Sagi
  }{%
    1 % number of printed copies
  }{%
    \today % date of completion
  }%
}{%department and address
  \textbf{Department of Computer Science}\\
  Selma Lagerløfs Vej 300\\
  \href{http://www.cs.aau.dk/}{http://www.cs.aau.dk/}
}{% the abstract
  This thesis explores data parsed~\cite{YaqutRB} from the medieval gazetteer,
  Kitāb Mu'jam al-Buldān written by Yâqût al-Hamawî~\cite{Yaqut}.

  The parsed dataset was created as part of The Middle East Heritage Data Integration Endeavour~\cite{MEHDIE}
  The MEHDIE Project attempts to aggregate multilingual historical information in the domain of
  Medieval Middle Eastern History.

  This thesis proposes an approach to transform the parsed Kitāb dataset into a knowledge graph to assist with MEHDIE's goal.
  To further assist the aggregation efforts, this thesis explores various link-prediction methods to predict new
  positive triplets and detect parsing errors.

  Moreover, it proposes a potential approach for generating a large and dense
  synthetic pre-training dataset for geospatial link prediction models using a Knowledge Graph representation
  of OpenStreetMaps~\cite{OpenStreetMap} called WorldKG~\cite{WorldKG}.

  Finally, it is demonstrated that a GNN model could act as an Error Detection model for the rule-based parser used
  to generate the original Kitāb dataset.
}
